%!TEX TS-program = xelatex

\documentclass[a4paper,14pt]{article}

\input{../text/preambular.tex}
\begin{document}
\begin{titlepage}
	\begin{center}
		ФЕДЕРАЛЬНОЕ  ГОСУДАРСТВЕННОЕ АВТОНОМНОЕ \\
		ОБРАЗОВАТЕЛЬНОЕ УЧРЕЖДЕНИЕ ВЫСШЕГО ОБРАЗОВАНИЯ\\
		«НАЦИОНАЛЬНЫЙ ИССЛЕДОВАТЕЛЬСКИЙ УНИВЕРСИТЕТ\\
		«ВЫСШАЯ ШКОЛА ЭКОНОМИКИ»
	\end{center}
	
	\begin{center}
		\textbf{Московский институт электроники и математики}
		
		\textbf{Им. А.Н.Тихонова НИУ ВШЭ}
		
		\textbf{Департамент компьютерной инженерии}
	\end{center}	
	\vspace{6ex}
	\begin{center}
	Отчёт по лабораторной работе \\
	<<Управление процессами в ОС UNIX>> \\ 
	 по курсу <<Операционные системы>>
	\end{center}	
	\vspace{5ex}
	
	Выполнил:
	
	Подчезерцев Алексей Евгеньевич 
	
	группа БИВ172
	
	\vspace{5ex}
	
	Принял:
	
	Фомин Сергей Сергеевич 
	
	\vspace{5ex}
	
	Оценка:


	\vfill
	\begin{center}
		Москва \the\year
	\end{center}
\end{titlepage}

\tableofcontents
\pagebreak

\section{Формулировка задания}
Разработать фильтр, выполняющий центрирование каждой строки вводимого текста (максимальная длина строки – 50 символов).
Придуманный фильтр -- перевести данные HEX формат.

\section{Конвейер двух фильтров}

\subsection{Исходный текст фильтра центрирования}
{\small \verbatiminput{center_filter.c}}
\subsection{Исходный текст фильтра перевода в HEX формат}
{\small \verbatiminput{hex_filter.c}}
\subsection{Исходный код программы}
{\small \verbatiminput{task1.c}}
\subsection{Вызов программы}
{\small \verbatiminput{task1.sh}}

\section{Применение неименованных программных каналов}

\subsection{Исходный код программы}
{\small \verbatiminput{task2.c}}
\subsection{Вызов программы}
{\small \verbatiminput{task2.sh}}

\section{Применение именованных программных каналов}

\subsection{Исходный код server.h}
{\small \verbatiminput{server.h}}
\subsection{Исходный код сервера}
{\small \verbatiminput{server.c}}
\subsection{Исходный код клиента}
{\small \verbatiminput{client.c}}
\subsection{Вызов программы}
{\small \verbatiminput{task3.sh}}

\end{document}