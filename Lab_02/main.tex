%!TEX TS-program = xelatex

\documentclass[a4paper,14pt]{article}

\input{../text/preambular.tex}
\begin{document}
\begin{titlepage}
	\begin{center}
		ФЕДЕРАЛЬНОЕ  ГОСУДАРСТВЕННОЕ АВТОНОМНОЕ \\
		ОБРАЗОВАТЕЛЬНОЕ УЧРЕЖДЕНИЕ ВЫСШЕГО ОБРАЗОВАНИЯ\\
		«НАЦИОНАЛЬНЫЙ ИССЛЕДОВАТЕЛЬСКИЙ УНИВЕРСИТЕТ\\
		«ВЫСШАЯ ШКОЛА ЭКОНОМИКИ»
	\end{center}
	
	\begin{center}
		\textbf{Московский институт электроники и математики}
		
		\textbf{Им. А.Н.Тихонова НИУ ВШЭ}
		
		\textbf{Департамент компьютерной инженерии}
	\end{center}	
	\vspace{6ex}
	\begin{center}
	Отчёт по лабораторной работе \\
	<<Управление процессами в ОС UNIX>> \\ 
	 по курсу <<Операционные системы>>
	\end{center}	
	\vspace{5ex}
	
	Выполнил:
	
	Подчезерцев Алексей Евгеньевич 
	
	группа БИВ172
	
	\vspace{5ex}
	
	Принял:
	
	Фомин Сергей Сергеевич 
	
	\vspace{5ex}
	
	Оценка:


	\vfill
	\begin{center}
		Москва \the\year
	\end{center}
\end{titlepage}

\begin{enumerate}
\item Ввести несколько переменных и присвоить им значения:
	{\small \verbatiminput{exec/1_create_vars.sh}}
\item просмотреть значения всех введенных переменных (встроенная команда set);
	{\small \verbatiminput{exec/2_var_list.sh}}
\item создать простейшую командную процедуру, в которой используется значение позиционных параметров;
	{\small \verbatiminput{exec/3_base_command.sh}}
\item ввести и отладить командные процедуры: print3, cmplist, copy3, menu;

	print3 source code
	{\small \verbatiminput{code/print3}}
	print3 execution
	{\small \verbatiminput{exec/4_print3.sh}}
	
	cmplist source code
	{\small \verbatiminput{code/cmplist}}
	cmplist execution
	{\small \verbatiminput{exec/4_cmplist.sh}}
	
	copy3 source code
	{\small \verbatiminput{code/copy3}}
	copy3 execution
	{\small \verbatiminput{exec/4_copy3.sh}}
	
	menu source code
	{\small \verbatiminput{code/menu}}
	menu execution
	{\small \verbatiminput{exec/4_menu.sh}}
\item модифицировать процедуру print3 так, чтобы она выводила на стандартный вывод только текстовые файлы;
	{\small \verbatiminput{code/print4}}
	{\small \verbatiminput{exec/5_print4.sh}}
\item модифицировать процедуру cmplist так, чтобы она правильно работала в случае, когда все сравниваемые файлы одинаковы;
	{\small \verbatiminput{code/cmplist2}}
	{\small \verbatiminput{exec/6_cmplist2.sh}}
\item модифицировать стандартные переменные sh (PS1, PS2, PATH);
	{\small \verbatiminput{exec/7_system.sh}}
\item проанализировать управляющий командный файл (.profile) в домашнем каталоге и каталоге /etc;
	{\small \verbatiminput{exec/8_profile.sh}}
	В файле устанавливаются программы для редактирования и просмотров документов, добавляются переменные из файла .shrc, отображается случайная подсказка для free-bsd
\item просмотреть стартовый командный файл /etc/rc и конфигурационный командный файл /etc/rc.conf;

	Файл /etc/rc устанавливает переменные, выполняет настройку, проверяет переменные.
	Файл /etc/rc.conf настраивает сетевое взаимодействие.
\item опробовать ввод командных процедур в интерактивном режиме.
	{\small \verbatiminput{exec/10_interactive.sh}}
\item Разработайте две командные процедуры.

	Находит все git репозитории в текущем поддереве и выполняет обновление каждого из них.
	{\small \verbatiminput{code/pull_all}}
	{\small \verbatiminput{exec/11_pull_all.sh}}
	
	Печатает только IP адрес, с которого осуществляется доступ к серверу.
	{\small \verbatiminput{code/who_ip}}
	{\small \verbatiminput{exec/11_who_ip.sh}}
\end{enumerate}


\end{document}