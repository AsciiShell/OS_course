%!TEX TS-program = xelatex

\documentclass[a4paper,14pt]{article}

\input{../text/preambular.tex}
\begin{document}
\begin{titlepage}
	\begin{center}
		ФЕДЕРАЛЬНОЕ  ГОСУДАРСТВЕННОЕ АВТОНОМНОЕ \\
		ОБРАЗОВАТЕЛЬНОЕ УЧРЕЖДЕНИЕ ВЫСШЕГО ОБРАЗОВАНИЯ\\
		«НАЦИОНАЛЬНЫЙ ИССЛЕДОВАТЕЛЬСКИЙ УНИВЕРСИТЕТ\\
		«ВЫСШАЯ ШКОЛА ЭКОНОМИКИ»
	\end{center}
	
	\begin{center}
		\textbf{Московский институт электроники и математики}
		
		\textbf{Им. А.Н.Тихонова НИУ ВШЭ}
		
		\textbf{Департамент компьютерной инженерии}
	\end{center}	
	\vspace{6ex}
	\begin{center}
	Отчёт по лабораторной работе \\
	<<Управление процессами в ОС UNIX>> \\ 
	 по курсу <<Операционные системы>>
	\end{center}	
	\vspace{5ex}
	
	Выполнил:
	
	Подчезерцев Алексей Евгеньевич 
	
	группа БИВ172
	
	\vspace{5ex}
	
	Принял:
	
	Фомин Сергей Сергеевич 
	
	\vspace{5ex}
	
	Оценка:


	\vfill
	\begin{center}
		Москва \the\year
	\end{center}
\end{titlepage}

\section{ЦЕЛИ РАБОТЫ}

\begin{enumerate}
\item Ввести несколько переменных и присвоить им значения:
	{\small \verbatiminput{code/create_vars.sh}}
\item просмотреть значения всех введенных переменных (встроенная команда set);
	{\small \verbatiminput{code/var_list.sh}}
\item создать простейшую командную процедуру, в которой используется значение позиционных параметров;
	{\small \verbatiminput{code/base_command.sh}}
\item ввести и отладить командные процедуры: print3, cmplist, copy3, menu;
\item модифицировать процедуру print3 так, чтобы она выводила на стандартный вывод только текстовые файлы;
\item модифицировать процедуру cmplist так, чтобы она правильно работала в случае, когда все сравниваемые файлы одинаковы;
\item модифицировать стандартные переменные sh (PS1, PS2, PATH);
\item проанализировать управляющий командный файл (.profile) в домашнем каталоге и каталоге /etc;
\item просмотреть стартовый командный файл /etc/rc и конфигурационный командный файл /etc/rc.conf;
\item опробовать ввод командных процедур в интерактивном режиме.

\end{enumerate}


\end{document}