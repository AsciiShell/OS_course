%!TEX TS-program = xelatex

\documentclass[a4paper,14pt]{article}

\input{../text/preambular.tex}
\begin{document}
\begin{titlepage}
	\begin{center}
		ФЕДЕРАЛЬНОЕ  ГОСУДАРСТВЕННОЕ АВТОНОМНОЕ \\
		ОБРАЗОВАТЕЛЬНОЕ УЧРЕЖДЕНИЕ ВЫСШЕГО ОБРАЗОВАНИЯ\\
		«НАЦИОНАЛЬНЫЙ ИССЛЕДОВАТЕЛЬСКИЙ УНИВЕРСИТЕТ\\
		«ВЫСШАЯ ШКОЛА ЭКОНОМИКИ»
	\end{center}
	
	\begin{center}
		\textbf{Московский институт электроники и математики}
		
		\textbf{Им. А.Н.Тихонова НИУ ВШЭ}
		
		\textbf{Департамент компьютерной инженерии}
	\end{center}	
	\vspace{6ex}
	\begin{center}
	Отчёт по лабораторной работе \\
	<<Управление процессами в ОС UNIX>> \\ 
	 по курсу <<Операционные системы>>
	\end{center}	
	\vspace{5ex}
	
	Выполнил:
	
	Подчезерцев Алексей Евгеньевич 
	
	группа БИВ172
	
	\vspace{5ex}
	
	Принял:
	
	Фомин Сергей Сергеевич 
	
	\vspace{5ex}
	
	Оценка:


	\vfill
	\begin{center}
		Москва \the\year
	\end{center}
\end{titlepage}

\tableofcontents
\pagebreak

\section{Задание 1}

\subsection{Формулировка задания}
Написать скрипт, который по выбору пользователя копирует, перемещает или удаляет заданный файл.

\subsection{Исходный текст процедуры}
{\small \verbatiminput{code/task_menu}}
\subsection{Вызов процедуры}
{\small \verbatiminput{exec/12_menu.sh}}
\subsection{Результат работы}
Была создана процедура, которая позволяет копировать, перемещать или удалять файл. 
Имя файла для обработки берется из первого аргумента, необходимое действие вводится с клавиатуры пользователем.
Если необходимо создать копию или переместить файл, будет запрошено имя нового файла.
При ошибке ввода команды пользователю будет выведено предупреждающее сообщение.

\section{Задание 2}

\subsection{Формулировка задания}
Разработать процедуру, которая выводит на экран по страницам все текстовые файлы.

\subsection{Исходный текст процедуры}
{\small \verbatiminput{code/task_print}}
\subsection{Вызов процедуры}
{\small \verbatiminput{exec/12_print.sh}}
\subsection{Результат работы}
Была создана процедура, которая печатает имена текстовых файлов и выводит постранично их на экран.
Для навигации по файлу можно использовать синтаксис команды $more$.


\end{document}